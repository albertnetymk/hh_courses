\documentclass{report}
\usepackage{syntonly}
\usepackage{verbatim}
\usepackage{array}
%\syntaxonly

\begin{document} 
\title{Project Plan} 
\maketitle 

\begin{table}[h]
\begin{tabular}{| l | l |}
	\hline
	Project name & Material handling \\
	\hline
	Client/Sponsor & StoraEnso plant in Hylte \\
	\hline
\end{tabular}
\end{table} 

\section{Executive summary}
Firstly, evaluate the data coming from the company production database, and reach the consensus how to deal with the invalid data. Develop one graphic 
user interface, using which users could easily understand the situation of the mill from different perspective.

\section{Background, objective and goal}
\begin{table}[h]
\begin{tabular}{| p{12cm} |}

	\hline
	Background \\
	\\
	Large quantities of paper are produced and shipped every day at a paper mill. Several trucks are involved to handle all these materials. 

	The situation is explained subsequently:
	\begin{enumerate}
	\item
	Paper reels are produced and come out on a conveyor belt.
	\item
	From conveyor belt they are transported to intermediate storage or to immediate shipping.
	\item
	From intermediate storage they are later transported to shipping (container, train, lorry).
	\end{enumerate}
		\\

	\hline
		Objective \\
		\\
		To develop a simulation environment to test techniques that can be used optimize the material handling. \\
	
	\hline
		Goal \\
		\\
		To convey information from the database, in order to do the further optimization.\\

	\hline
		Limitation \\
		\\
		Possible optimization approach will not be proposed or simulated \\

	\hline

\end{tabular}
\end{table}

\section{Requirement specification}
Requirement form \\
\begin{table}[h]
\begin{tabular}{| p{12cm} |}
	\hline
	Product requirement \\
	\begin{itemize}
	\item
	The situation expressed in the form of database is presented in the form that is easily understandable for human.
	\end{itemize}
	\\

	\hline
	Project requirement \\
	\begin{itemize}
	%\item
	%The simulation environment must have a graphic interface, where different suggested solutions or improvements can be evaluated.
	\item
	The simulation environment must be a copy of the StoraEnso plant in Hylte.
	\item
	The simulation environment must have the function to extracting data from the StoraEnso production system database.
	\item
	The simulation environment must have the ability to present the user with the total distance based on each truck and all trucks.
	\item
	The simulation environment must have the information concerning the usage of all cells.
	\item
	The simulation environment should have the ability to illustrate the production flow at the StoraEnso plant.
	\item
	The function of extracting data from StoraEnso production system database could be a software.
	\end{itemize}
	\\
	\hline
	Prerequisites \\
	\begin{itemize}
	\item
	The information about StoraEnso production system database, such as how the data is gathered, and the platform this database is using.
	\item
	Periodic reference must be provided during the execution of this project.
	\end{itemize}
	\\
	\hline
\end{tabular}
\end{table}

\section{Milestones, activities and schedule}
\subsection{Milestone plan}
	\begin{itemize}
	\item
	Construct the distance matrix from the plant map given that all trucks are moving between 2 cells directly.
	\item
	Obtain the valid data from the raw data.
	\item
	Get the statistical information of usage of cells.
	\item
	Construct the map of the mill.
	\item
	Get the statistical information of trucks.
	\item
	Associate all information obtained before with the map.
	\end{itemize}
	
\subsection{Activity list}
	\begin{table}[h]
	\begin{tabular}{|m{0.1cm} | m{5cm} | m{5cm} | m{1.5cm} | m{1.5cm} |}
	\hline
	1 & Evaluate the data in the database, and determine how to deal with invalid data  & Project sponsor and engineers from Hylte & May 17th 10:00 AM 
	& May 17th 12:00 PM \\
	\hline
	%2 & Design the \hline
	2 & Obtain valid data after preprocessing the raw data from database & Project members & May 18th & May 31th \\
	\hline
	3 & Gather the information from the parsed data and extract the statistic information about usage of cell & Project members & June 1th & June 7th 
	\\
	\hline
	4 & Construct the map of the mill & Project members & May 17th & June 18th \\
	\hline
	5 & Illustrate the production flow in the simulation environment & Project members & June 19th & June 29th \\
	\hline
	6 & Construct the distance matrix & Project members & May 17th & May 31th \\
	\hline
	
	\end{tabular}
	\end{table}

\subsection{Gantt chart}
Please see the open office file.

\end{document}
