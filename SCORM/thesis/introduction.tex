\chapter{Introduction}
\thispagestyle{empty}
In this chapter, introduction to this project is covered, including background, objectives, limitations, etc.
\section{Background}
Essentially, E-learning is the transferring of skills and knowledge through computer and networks. With the rapid development of Internet, E-learning 
is becoming more and more popular and practical. E-learning applications include Web-based learning, computer-based learning, virtual classroom 
opportunities and digital collaboration. This project is proposed by Entergate company (www.entergate.se), and used to extend the original 
functionality of their product ``esTracer'', which is the Learning Management System from the company.
\section{Related work}
When it comes to E-learning, Advanced Distributed Learning has to be mentioned. The Advanced Distributed Learning (ADL) Initiative is the result of 
collaborative efforts to harness the power of information technologies to deliver high-quality, easily accessible, adaptable, and cost-effective 
education and training \cite{adl_intro}. More information about ADL will be covered when the origin of Sharable Content Object Reference Model 
(SCORM), which is one standard related to E-learning and will be explained in detail in next chapter, is addressed.

The most prestigious company providing SCORM related solution is Rustici Software \cite{rustici_software}. They make SCORM easy by creating the best 
SCORM products on the market. In fact, ADL is sponsoring them to conduct the Tin Can research, which is meant to be the successor of SCORM. Further 
information can be found in section~\ref{sec:scorm_variation}.
\section{Objectives}
Without following the standards, endless of time and money will be wasted for ``reinventing the wheel''. Considering SCORM is the de facto standard 
in E-learning system, the objective of this project is to make ``esTracer'' SCORM compliant. This project focuses on importing and exporting SCORM 
compliant packages, and present the content of packages without providing communication with the Learning Management System(LMS). This project is one 
preliminary work of making ``esTracer'' SCORM compliant.

The whole mission is broken into four tasks:
\begin{itemize}
	\item Importing whole package \\
		This is used when the whole packaged is meant to be launched by LMS directly.
	\item Importing all SCOs in this package \\
		This will enable course content authors to create complete tests out of all the SCOs, which could then be exported as SCORM compliant 
		packages or launched by LMS.
	\item Exporting package \\
		This could be used when different LMSs want to share packages.
	\item Launching SCO \\
		Via this, users can interact with the SCO. (In this project, the communication between SCO and LMS is not implemented, so no records will
		be save after finishing this SCO.)
\end{itemize}
\section{Limitation}
Considering E-learning is one quite broad concept, this project will only concern about Web-based learning E-learning system, specifically the 
product from Entergate company. The implementation only applies to ``esTracer'', but the methodology is general, which can be used in other systems.

Make one Learning Management System SCORM compliant requires launching the content in one frameset or one new windows, then to provide the APIs to 
enable the communication between the Learning Management System and the content. This project only covers the importing, exporting and previewing the 
content of SCO without providing the communication APIs.
\section{Outline}
In Web-based E-learning System and SCORM chapter, E-learning and SCORM is addressed to give the audiences the fundamental concepts concerning E-learning 
and the de facto standard. The standard is explained thoroughly because this project depends on it heavily.

In the Method chapter, the working environment and methodology adopted is presented in detail. The ``esTracer'' is explained from a technical perspective
so that modification could be made in order to extend its functionality. Afterwards, how each task is solved, and why one particular approach is 
chosen are explained.

In the Result chapter, one package from ADL is used to demonstrate the use of functionality concerning SCORM in ``esTracer'', such as importing and 
presentation.

In the Discussion and Conclusion chapter, personal opinion about various technology used in this project is expressed. Since this project is just one 
preliminary task of making ``esTracer'' SCORM compliant, future work would be done, some of which are discussed.
