\documentclass{report}
\usepackage[left=3cm]{geometry}
\usepackage{syntonly}
\usepackage{verbatim}
\usepackage{array}
\usepackage{tabularx}
\setlength{\extrarowheight}{5pt}
%\syntaxonly

\begin{document} 
\title{Project Plan} 
\maketitle 

\begin{table}[h]
	\begin{tabularx}{\textwidth}{| l | X |}
	\hline
	Project name & Communication between various e-learning system using SCORM \\
	\hline
	Client/Sponsor & Entergate Company \\
	\hline
\end{tabularx}
\end{table} 

\section{Executive summary}
Firstly, do some theoretical research on SCORM and related information. Secondly, understand the internal structure of one Package Interchange 
File(PIF). Then, extract related information form it, and integrate them into the LMS in Entergate.

\section{Background, objective and goal}
\begin{table}[h]
	\begin{tabularx}{\textwidth}{| X |}
	\hline
	Background \\
	\\
	In the late 1990s, the United States Department of Defense realized that they were procuring the same training many times over and over again but 
	could not reuse it across departments, because each department had its own Learning Management System(LMS). In those days, each LMS had its own 
	proprietary content format, which encouraged vendor lock-in. In 1999 an executive order tasked a small research laboratory, Advanced Distributed 
	Learning(ADL), to ``develop common specifications and standards for e-learning.'' Rather than starting from scratch, ADL harmonized the work of 
	existing standards organizations like the AICC, IMS and the IEEE LTSC into a cohesive reference model. SCORM was released in 2001 and was quickly 
	adopted by both government and industry. Today it is the de facto standard for E-learning interoperability. \\ [1ex]

	\hline
		Objective \\
		\\
		Utilizing the existing SCORM standard to achieve the communication between different LMSs. \\ [1ex]
	
	\hline
		Goal \\
		\\
		SCORM compliant contents can be imported smoothly into our LMS. \\  [1ex]
%	\hline
%		Limitation \\
%		\\
	\hline
\end{tabularx}
\end{table}

\newpage
\section{Requirement specification}
Requirement form \\
\begin{table}[h]
	\begin{tabular}{| p{15cm} |}
		\hline
		Product requirement \\
		\begin{itemize}
			\item
				Compare the pros and cons of different techniques enabling communication between different LMSs, and decide which one is the best in 
				this case.
			\item
				The learning contents from one LMS must be used in our LMS, smoothly.
			\item
				In the first case, the solution must enable that one whole package, which contains some preexistent questions, can be used as one test
				for users to take.
			\item
				In the second case, the solution must enable that several questions in this package, combined with our questions, can constitute the 
				test for users to take.
			\item
				The solution could have the ability to present the information we imported to users.
		\end{itemize}
		\\

		\hline
		Project requirement \\
		\begin{itemize}
			\item
				Finished before May 1st.
			\item
				The method should be based on SCORM.
		\end{itemize}
		\\
		\hline
		Prerequisites \\
		\begin{itemize}
			\item
				Detailed description on LMS in Entergate.
			\item
				Constant consultancies on SCORM and LMS.
		\end{itemize}
		\\
		\hline
	\end{tabular}
\end{table}

\section{Milestones, activities and schedule}
\subsection{Milestone plan}
\begin{itemize}
	\item
		Be able to import Manifest Basics Content Example(MBCE) from ADL to esTracer.
	\item
		Present the information we get from this package to users in our LMS.
\end{itemize}
\subsection{Activity list}
\begin{table}[h]
	\begin{tabular}{|m{0.1cm} | m{5cm} | m{5cm} | m{2.5cm} | m{2.5cm} |}
		\hline
		1 & Theoretical research on SCORM, especially Content packaging & Project members & February 10th & February 28th \\
		\hline
		2 & Understand jQuery and MVC, which are used in LMS & Project members & February 10th & February 28th \\
		\hline
		3 & Understand internal structure of LMS in Entergate & Entergate staff and project members & February 17th & March 15th \\
		\hline
		4 & Implement import function & Entergate staff and project members & March 15th & March 31st \\
		\hline
		4 & Implement present function & Entergate staff and project members & April 1st & April 30th \\
		\hline
		5 & Writing the final report & Entergate staff and project members & May 1st & May 7th \\
		\hline
	\end{tabular}
\end{table}
\end{document}
